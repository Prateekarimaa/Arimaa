There will always be problems or delays that come up. The best way to deal with them is to think about what could go wrong in the project rather than waiting to get into deep water. The five kinds of risks are the following :
\begin{enumerate}
	\item technical
	\item Ressources
	\item Organisation
	\item Payments
	\item Suppliers/Purchases
\end{enumerate}
Our main risk factor is the point number 1 : technical. We are not dependant to anything like purchases, ressources or organisation as we are a small group of workers and we barely need anything but our computer.\\
To get into further details, here is a list of what might possibly become cumbersome :
\begin{itemize}
	\item getting used to the technology we will use (CAF, OpenMPI, OpenAcc, OpenMP, Boost Library...)
	\item booking the use of Grid5000, we don't know how long is the waiting list in order to be able to borrow it.
	\item Interoperability problems : most of us works on Windows Operating Systems. However Grid5000 runs on Linux, for this reason we will test our algorithm on a smaller scale. The computer science departement at INSA will be our first testing facility in order to make our cluster implemtation works.
\end{itemize}
Some other problems might come up later and we will try to deal with them as soon as possible to get the least delay as we possibly could.~\\
