\subsubsection{Tasks gathered in modules}

To understand better how the tasks are organized, tasks have been classified in modules, representing the main aspects of our project. Each task will be checked, and tested just before the end of the test, according to the Agile method.

\begin{itemize}
  \item The first module is the GUI application. It contains a Graphical Interface for our game, but as well the rules to play, and a recorder of rules. 
This application will make it possible for an human to play against another human.
  \item The second module is the MCTS algorithm. It is the heart of our project. This algorithm will be used to compute a set of all moves possible ordered in a tree. First, it will be implemented on \textit{Tic-Tac-Toe}, then on \textit{Connect4} and finally on \textit{Arimaa}. 

With this module, it will be possible to play against the AI.
There will be as well some improvements to do as the use of Boost library, a better memory management, and making statistics.
  \item The module Converter is a converter of data to make communicate the MCTS part and the GUI part.
  \item The Parallelization module represents the management of our environment. Our algorithm will be tested using different computer powers, with CPU and GPU parallelization. It will be tested on our computers, on the computers of our INSA computer science department. Finally, if the time permits us, we will test it on the researchers' network \textit{Grid'5000}. 

This module will increase the performances of our algorithm, giving it calculus power.
  \item The last module is the Documentation. It will take a lot of our time, writing 2 reports, preparing 2 oral presentations, conceiving an HTML page and a CD of the application.
\end{itemize}
All these tasks form part of the Gantt diagram.

\subsubsection{Gantt diagram}
TODO baptiste