The project is called \emph{Fast \& Furious Game Playing, Monte Carlo Drift}. Its purpose is to create an Artificial Intelligence able to compete against humans using the \emph{Monte Carlo Tree Search} algorithm.

We have chosen the game Arimaa because it is a two-players strategy board game not solved\footnote{A game solved is a game where good algorithms are able to find the perfect move in each situation to win, or to draw. For instance, \textit{Tic Tac Toe} or \textit{Draughts} are solved games.}.

A human plays a game by thinking of all possible moves as per one's imagination and then opts for the best amongst them. Before taking ones turn, a player can visualize ones options and predict how an opponent will counteract them. The algorithm will do the same by building a search tree containing the different posibilities. The Minimax algorithm does it by exploring all possibilities, which is heavy. The MCTS algorithm is lighter and converges to the Minimax algorithm, therefore it has been chosen for this project.

This algorithm will be parallelized in order to optimize it in a set of multi-core machines, allowing it to go further into the search tree, thus improving its efficiency.

The planning methods we used will be explained. Furthermore, we will develop the context of our environment namely we will precise the main dates on our calendar and the ressources we will have.  Then, we will plan the different tasks we will realise along the project. With those tasks and the dependencies between them, we will create a Gantt diagram in order to graphically represent our planning. In order to reduce the brakes on the project, we will analyse the different risks about it.
