%this file is the second report
%a % comment anything after % until the end of the line

%minimum references to begin our article
\documentclass[12pt]{article}
\usepackage[english]{babel}
\usepackage[utf8]{inputenc}
\usepackage[T1]{fontenc}
\usepackage{graphicx}
\usepackage{fancyhdr}
\usepackage{hyperref}
\usepackage{float}
\usepackage{amsmath}
\usepackage[margin=1in]{geometry}
\usepackage{indentfirst}
\usepackage{titlesec}
\usepackage{caption}
\newcommand{\sectionbreak}{\clearpage}

\pagestyle{fancy}
%\cfoot{Fast and furious game playing: Monte Carlo drift}
% the last extension makes it possible to add images

%presentation of the document
\title{Fast and Furious Game Playing : Monte Carlo Drift\smallbreak Specifications report} %not sure about the name of this report
\author{Prateek \textsc{Bhatnagar}, Baptiste \textsc{Bignon}, \\
		Mikaïl \textsc{Demirdelen}, Gabriel \textsc{Prevosto}, \\
		Dan \textsc{Seeruttun-{}-Marie}, Benoît \textsc{Viguier} \\
		\\
		Supervisors: Nikolaos \textsc{Parlavantzas}, Christian \textsc{Raymond}}
\date{11/27/2014}
\setlength\parindent{15pt}
\begin{document}
\maketitle

The MCTS algorithm has been chosen in order to develop our artificial intelligence. However, in order to improve its efficiency, we will need to adapt it. The main problem is the branching factor\footnote{In a tree,  the branching factor is the number of children at each node.} of the Arimaa game which average is 17 281 and reaches about 22 000 after 10 moves\footnote{http://arimaa.janzert.com/bf\_study/}.
\bigskip
\begin{center}
	\begin{tabular}{ | c | c |}
		\hline Game & Average number of possible moves \\ \hline
		\hline  
		Othello & 8\\
		\hline  
		Chess & 35\\
		\hline  
		Game of Go & 250\\
		\hline
		Arimaa & 17 281\\
		\hline
	\end{tabular}
\end{center}
\bigskip
The reason why the branching factor of a game is so important is because it increases greatly the space that has to be searched in order to guess what will happend multiples moves ahead. In chess after 6 moves, the number of positions evaluated are about 35\textsuperscript{6} which is roughtly equivalent to 1,8 billions. In Arimaa, after 3 turns (yours, the opponent and yours again), if you were the explore all positions, you would need to evaluate around 5,2 trillions\footnote{1 trillion = 1 thousand billions = 10\textsuperscript{12}.} boards (2000 times more than chess with half the number of moves).
\bigskip\\
In order to decrease the space to be search, our MCTS Algorithm will perform a big number of simulations before chosing the nodes to explore. After the selection, it will prune the tree in order to optimise search speed and the memory managment.

\end{document}
