During our research for GPU\footnote{Graphic Processing Unit} parallelization, three possibilies have been found:
\begin{itemize}
\item OpenMp
\item OpenACC
\item CUDA
\end{itemize}


\subsubsection{OpenMP}
OpenMP supports the GPU programming since its version 4.0, it implements some of the methods of OpenACC. But it seems to be less efficient and complete than OpenAcc so we decided to avoid to use it.

\subsubsection{CUDA}
CUDA is a framework develop by NVIDIA dedicated to the use of GPU for complex calculation. It allows very efficient and low-level computations but it forces to rewrite all the parallel code. Its means that we will need one code for machines without GPU and one for machines with GPU. Furthermore, the product code is illegible for people who do not master CUDA.

\subsubsection{OpenACC}
OpenACC is an API created by the authors of OpenMP to implement the GPU computation before to integrate it in OpenMP. A part of its fonctionnalities had been added in OpenMP 4.0 but OpenACC is still more complete and efficient, especially for the data management. This point is very important, because if it is not correctly managed, there will be too many transfer of data between the CPU and GPU. In this case we can lose more time than we gained on the calculation. 

As we consider OpenACC more simple, and easier to maintain for just a little drop of performance compared to CUDA, we chose it to design our software to take advantage of the presence of GPU.