Our last software need is about the cluster parallelization. As we have decided to use a Root Parallelization there won't be many communication between the different computers, we could choose between two solutions : the Sockets and MPI.

\subsubsection{Sockets}

A socket is used to communicate across a computer network. The socket is an end point of the communication flow. A socket is a low-level mecanism.

\subsubsection{MPI}

MPI, which means Message Passing Interface, is a standardized message-passing system. It allows us to communicate between computers which belong to a network by sending messages between them. 

\subsubsection{The chosen solution}

We decided to use MPI for many reasons.
\begin{itemize}
\item MPI is more high-level than the sockets. So, it will be simpler to implement in our software.
\item The community behind MPI is large so there is more documentation about MPI than the sockets. It would be simpler to fix the different problems.
\item The operation of MPI is based on the sockets so it is similar to the sockets, in terms of performances.
\end{itemize}
In conclusion, MPI would be simpler to implement, more documented than the sockets while they both have the same performances.

\subsubsection{Basic operations}

The basic operations of MPI are quite simple. In the main, we first have to initialize MPI with this function :\\\\
MPI\_Init();\\\\
After the initialization, we have to precise how many computers there are in the network with :\\\\
int MPI\_Comm\_size(MPI\_Comm comm, int *size);\\\\
We can also define individually the 'name', or rank, of each computers in the network with :\\\\
int MPI\_Comm\_rank(MPI\_Comm comm, int *rank);\\\\
Then we can send and receive messages with those functions, respectively :\\\\
MPI\_Send();\\
MPI\_Recv();\\\\
When we have finsh to use MPI, we can terminate it properly :\\\\
MPI\_Finalize();\\