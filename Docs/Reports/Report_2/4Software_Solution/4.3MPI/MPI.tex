Our last software need is about cluster parallelization. As we have decided to use Root Parallelization, there will not be a lot of communication between the different computers. We can choose between two solutions : the Sockets and MPI.
\begin{itemize}
\item A socket is  a low-level mechanism used to communicate across a computer network. The socket is an end point of the communication flow.
\item MPI is a standardized message-passing system. It allows us to communicate between computers which belong to a network by sending messages between them. 
\end{itemize}
\subsubsection{The chosen solution}

We decided to use MPI for the following reasons:
\begin{itemize}
\item MPI is more high-level than the sockets, thus it will be simpler to implement in our software.
\item The community behind MPI is large so there wouldn't be any problem to fix the different bugs. Moreover, the MPI documentation is clearer than the sockets'. 
\item MPI uses sockets so it is similar, though a little bit inferior to the sockets, in term of performance.
\end{itemize}
In conclusion, MPI would be simpler to implement, is more documented than the sockets while they both have almost similar performance. So, it seems more adapted to our project than the sockets.