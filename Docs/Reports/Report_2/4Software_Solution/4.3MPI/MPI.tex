Our last software need is about the cluster parallelization. As we have decided to use a Root Parallelization there will not be a lot of communication between different computers, we could choose between two solutions: the Sockets and MPI.
\begin{itemize}
\item A socket is used to communicate across a computer network. The socket is an end point of the communication flow. A socket is a low-level mechanism.
\item MPI is a standardized message-passing system. It allows us to communicate between computers which belong to a network by sending messages between them. 
\end{itemize}
\subsubsection{The chosen solution}

To compare them, we can see that :
\begin{itemize}
\item MPI is more high-level than the sockets. So, it will be simpler to implement in our software.
\item The community behind MPI is large so there wouldn't be any problem to fix the different bugs. Moreover, the MPI documentation is clearer than the sockets' one. 
\item The operation of MPI is based on the sockets so it is similar, though a little bit inferior, to the sockets, in term of performances.
\end{itemize}
In conclusion, MPI would be simpler to implement, more documented than the sockets while they both have almost similar performance. So, it seems more adapted to our project than the sockets.