The project is called \emph{Fast \& Furious Game Playing, Monte Carlo Drift}. Its purpose is to create an Artificial Intelligence able to compete against humans using the \emph{Monte Carlo Tree Search} algorithm.

We have chosen the game Arimaa because it is a two-players strategy board game not solved\footnote{A game solved is a game where good algorithms are able to find the perfect move in each situation to win, or to draw. For instance, \textit{Tic Tac Toe} or \textit{Draughts} are solved games.}.

A human plays a game by thinking of all possible moves as per one's imagination and then opts for the best amongst them. Before taking ones turn, a player can visualize ones options and predict how an opponent will counteract them. The algorithm will do the same by building a search tree containing the different posibilities. The Minimax algorithm does it by exploring all possibilities, which is heavy. The MCTS algorithm is lighter and converges to the Minimax algorithm, therefore it has been chosen for this project.

This algorithm will be parallelized in order to optimize it in a set of multi-core machines, allowing it to go further into the search tree, thus improving its efficiency.

In this report, the data structures that will be used for the development of the project are discussed.
A parallelization strategy will be finalised from the various strategies that have been discussed till now by referring the test results. Various parallelisation frameworks viz. OpenMP, MPI and CAF will be tested and the most optimal framework for the project will be chosen for implementation of the project.
A study for the GUI of the project is done with the help of UML diagrams including the detailed explanation about the various interfaces that will be used.