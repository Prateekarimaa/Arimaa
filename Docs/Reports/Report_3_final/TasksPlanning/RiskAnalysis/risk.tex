Problems or delays are inevitable. The best way to deal with them is to predict what could go wrong in the project rather than waiting to get into deep water. The 5 categories below are those in which risk is most likely to occur :
\begin{enumerate}
	\item Technical
	\item Resources
	\item Organisation
	\item Payments
	\item Suppliers/Purchases
\end{enumerate}
The main risk factor is the first item : technical. We are not dependent on anything such as purchases, ressources or organisation as we are a small group of workers.

A list of what might possibly become cumbersome may now be established 	:
\begin{itemize}
	\item getting used to the technology we will use (CAF, OpenMPI, OpenACC, OpenMP, Boost Library...).
	\item unexpected bugs in the program.
	\item booking the use of \textit{Grid'5000}.
	\item interoperability problems : most of us work on \textit{Windows} operating systems. However \textit{Grid'5000} runs on Linux, for this reason we will test our algorithm on a smaller scale. Our cluster implementation will first be put to the test at the Computer Science department at INSA.
\end{itemize}
In order to avoid theses difficulties, we made sure to dedicate enough time to study the technology used and to fix the bugs. We will also make sure to book \textit{Grid'5000} early. Were it not be avialable, our tests would be done on a set of clusters from INSA's Computer Science department. While the computing power is not comparable to \textit{Grid'5000}, we still expect to get reliable results. A monthly test of our implementations in the Linux rooms of the department will guarantee us that our application does not have any interoperability problems.

Some other problems might come up later and we will try to deal with them as soon as possible to avoid unnecessary delays.~\\
