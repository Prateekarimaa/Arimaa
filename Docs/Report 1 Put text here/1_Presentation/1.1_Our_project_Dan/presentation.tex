
In 1997, Deep Blue, a supercomputer built by IBM, won a six games match against Garry Kasparov, the current world chess champion. Humans got beaten in Chess, but remain undefeated in other games. Consequently, researchers are looking for improvements in Artificial Intelligence.
\newline

The project is called \emph{Fast \& Furious Game Playing, Monte Carlo Drift}. Its purpose is to create an Artificial Intelligence able to compete against humans using the \emph{Monte Carlo Tree Search}.
\newline

The focus of this project is on two player strategy board games, while avoiding games already solved\footnote{A game solved is a game where good algorithms are able to find the perfect move in each situation to win, or to draw. For instance, \textit{Tic Tac Toe} or \textit{Draughts} are solved games.}. That is why this project is about the game \emph{Arimaa}.
\newline

A human plays a game by thinking of all possible moves as per one's imagination and then opts the best amongst them. The player can be able to look further for the opponent's turn, and it expands possibilities of moves. The Minimax algorithm does exactly the same thing, but it is exploring all possibilities. The current position is represented by a node, and the possible states after a move are represented by other nodes, pointing to the parent node. Then it forms a tree.
Minimax algorithm develops a tree by creating nodes for all possible moves.
\newline

The problem is to compute this heavy algorithm. The \emph{MCTS} algorithm has been conceived to choose  the nodes it wants to develop, so it is used for the project.
The \emph{Monte Carlo Tree Search} has been used in the past for \textit{Draughts}. By exploring numerous random possibilities, it will be able to take decisions in order to win the game.
The algorithm would be parallelized in order to exploit it in a multi-core machine, allowing it to go further into the search tree, thus improving its efficiency.
\newline

Different parallelization methods will be studied to choose the most suitable for this project.
The exploration of the tree will depend on the parallelization method.
The initial phase of the project would be the analysis of the latest papers concerning the technologies that might be of use.
The consecutive phases will be about making a choice among these technologies, to decide on the specifications.
Finally, in the last phase, a solution will be implemented, and executed on Grid'5000, on a set of cluster of multi-core machines.
\newline

In this report, the first part is about the presentation of Arimaa (see part \ref{first part}.) Then, different algorithms about the game are introduced (see part \ref{second part}). The third part is dedicated to parallelization methods (see part \ref{third part}) and the state of the art (see part \ref{third part part}). In the end, solutions (see part \ref{last part}) and the schedule is presented (see part \ref{last last part}).
The interesting part about this project is the creation of an artificial intelligence as optimized as possible.