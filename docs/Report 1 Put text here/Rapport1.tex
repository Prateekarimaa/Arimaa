%this file is the first report
%a % comment anything after % until the end of the line

%minimum references to begin our article
\documentclass[12pt]{article}
\usepackage[english]{babel}
\usepackage[utf8]{inputenc}
\usepackage[T1]{fontenc}
\usepackage{graphicx}
\usepackage{fancyhdr}
\pagestyle{fancy}
\cfoot{Fast and furious game : MonteCarlo drift}
% the last extension makes it possible to add images

%presentation of the document
\title{Fast and furious game : MonteCarlo drift\smallbreak Pre-study and analysis report}
\author{Prateek Bhatnagar, Baptiste Bignon, Mikaïl Demirdelen,\endline Gabriel Prevosto, Dan Seeruttun--Marie,  Benoît Viguier}
\date{10/15/2014}
\begin{document}
\maketitle
\includegraphics[scale=0.5]{img/arimaa}
\newpage
\begin{abstract}
Our project is called Fast and furious game playing, MonteCarlo drift. Our purpose is to create an Artificial Intelligence able to compete against humans using the MonteCarlo Tree Research.
\newline
We will only focus on two players games. Furthermore, we want to avoid games already resolved. We will choose something not studied entirely. We want to work on some new application. That is why we are interested by Arimaa.
\smallbreak
For our game, we will need a program and statistics to make a good Artificial Intelligence. Each move should be calculated using a reliable method.
MonteCarlo Tree Research is an algorithm able to take these optimal decisions. It has been used in the past for draughts, or chess. By exploring numerous possibilities, it will become possible to know what move is the better one.
We will parallelize this algorithm in order to use it in a multi-core machine, to improve his efficiency.
That MCTS algorithm is better than the classic Min-Max algorithm, that is why we will use it.
\smallbreak
We will analyse parallelization methods, we will present it, and then we will choose the one adapted to our project.
Thanks to the results of these latest methods, we will be able to choose a state resulting of the current move. Then we will explore the tree and with the same methods as before, we will figure out what the opponent will most probably do. The way we will be exploring the tree will only depend on the parallelization method.
The first part of our project will be the analysis of latest thesis of technologies we will use, in order to choose the best one, and using it on the right environment, to improve his  efficiency.
In the next part, we will choose technologies we will need to achieve our goals, we will create a UML diagram to settle down our program.
\smallbreak
Finally, in the last part, we will implement this program, and its documentation and test his executing on Grid5000, a cluster of multi-core machines.
What is interesting in this project is we will create an Artificial Intelligence using technologies and methods fully optimized. Then we will create a program that can lead to true improvements for current algorithms applied to this game.
\end{abstract}


\newpage
%to add a table of contents
\tableofcontents
\newpage

%to create subsections and subsubsubsections, we can use : (chapter n'existe pas avec la classe article)
\section{Presentation of our project}
\subsection{Generalities}
Insert Text Here
\subsection{Algorithm MCTS}
Insert Text Here
\subsection{Presentation of Arimaa}
Insert Text Here

\newpage
\section{Strategies and state of the art}
\subsection{Strategy of root parallelization}
Insert Text Here
\subsection{State of the art}
\subsubsection{Arimaa}
Insert Text Here
\subsubsection{MCTS}
Insert Text Here

\newpage
\section{Solutions and schedule of our project}
\subsection{Solutions we could use}
Insert Text Here
\subsection {Tasks' schedule}
Insert Text Here
\newpage
\section{Conclusion}
Finally, our project is about creating an Artificial Intelligence able to beat an human playing the Arimaa game.
\newline
\newline
Arimaa is a two players game. It has been designed to be difficult to foresee for computers, but easy to play for humans. 
In order to realize our project, we will need some concepts and technologies. 
We will use the MonteCarlo Tree Research algorithm, to take the best decisions in our game. We will decide what move to play according to this algorithm figures. Because of the numerous moves possible, we will need to choose to develop the better ones, that will depend on the chosen parallelization tree. Any variation could totally change statistics.
We already analyse the state of the art of Arimaa and the MonteCarlo Tree Research. Consequently, we will base our work on these thesis, to make it possible to use the best technologies with the best environment without doing what has already been done.
We already know how to play this game. So it would be easier to create strategies for our Artificial Intelligence. We handle a task's schedule to control our requirements. We will be able to give back our work to the due time. The point of this part is mostly to learn to schedule a project, foreseeing surprising events, and answering the demand.
\smallbreak
The point of this project is as well to test our program upon Grid5000, a powerful network of multi-core machines. Then, we will use the entire potential of our program in his maximal capacities, against a human.


\end{document}